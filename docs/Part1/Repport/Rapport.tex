\documentclass[11pt,a4paper]{article}

\usepackage[T1]{fontenc}
\usepackage[utf8]{inputenc}
\usepackage[french]{babel}
\usepackage{lmodern}
\usepackage{microtype}
\usepackage{geometry}
\geometry{margin=2.5cm}
\usepackage{hyperref}
\usepackage{csquotes}
\usepackage{graphicx}
\usepackage{amsmath}
\usepackage{siunitx}
\usepackage{listings}
\usepackage{xcolor}
\usepackage{tikz}
\usetikzlibrary{arrows.meta,positioning}

\lstset{
  basicstyle=\ttfamily\small,
  breaklines=true,
  columns=fullflexible,
  frame=single,
  rulecolor=\color{black},
  showstringspaces=false
}

\title{part 1 --- Scénario minimal Wi-Fi sous ns-3\\(1 AP, 1 STA, Ping et traces PCAP)}
\author{Jafarizadeh Mehdi}
\date{\today}

\begin{document}
\maketitle

\begin{abstract}
Ce rapport documente la conception et la validation d’un scénario Wi-Fi minimal sous ns-3, conforme aux exigences du part~1 :
une topologie composée d’un point d’accès (AP) et d’une station (STA), adressage IPv4, test de connectivité via \texttt{ping} (ICMP),
génération de traces PCAP et organisation reproductible des sorties. L’objectif n’est pas de mesurer des performances,
mais d’établir une base expérimentale robuste, paramétrable et vérifiable, réutilisable pour des scénarios plus avancés.
\end{abstract}

\section{Contexte et objectifs}
Un scénario minimal bien validé est indispensable avant toute étude WLAN : il permet de vérifier, de bout en bout, la chaîne
\emph{association Wi-Fi} $\rightarrow$ \emph{pile IP} $\rightarrow$ \emph{application}. Dans ns-3, un échec apparent (pas de réponses ICMP)
peut provenir d’un mauvais SSID, d’un démarrage applicatif trop précoce, d’un adressage incohérent ou d’une sortie non collectée.
Le part~1 impose ainsi un socle méthodologique : paramétrage en ligne de commande, production de preuves (ping et PCAP) et
stockage standardisé des résultats pour assurer la reproductibilité.

\section{Environnement expérimental}
\subsection{Système et version de ns-3}
Les expériences ont été réalisées avec :
\begin{itemize}
  \item OS : Ubuntu 24.04.3 LTS (\texttt{noble}).
  \item ns-3 : \texttt{ns-3.46.1-128-g94b49bb34}.
\end{itemize}
Dans cette distribution ns-3, les arguments de l’exécutable doivent être passés après \texttt{--} dans la commande \texttt{./ns3 run},
ce qui est intégré dans la procédure d’exécution.

\section{Description du scénario}
\subsection{Topologie et mobilité}
Le scénario comprend exactement deux nœuds :
\begin{itemize}
  \item un AP en position fixe $(0,0,0)$,
  \item une STA en position fixe $(d,0,0)$, où $d$ est paramétrable.
\end{itemize}

\begin{figure}[h]
\centering
\begin{tikzpicture}[
  node/.style={draw,rounded corners,minimum width=3.4cm,minimum height=1cm,align=center},
  link/.style={-Latex,thick}
]
\node[node] (ap) {AP\\\texttt{ApWifiMac}};
\node[node,right=4.3cm of ap] (sta) {STA\\\texttt{StaWifiMac}};
\draw[link] (sta) -- node[above]{Wi-Fi (SSID)} (ap);
\node[below=0.5cm of ap] {\small $(0,0,0)$};
\node[below=0.5cm of sta] {\small $(d,0,0)$};
\end{tikzpicture}
\caption{Topologie minimale : un AP et une STA séparés par une distance $d$.}
\end{figure}

\subsection{Pile réseau et preuve de fonctionnement}
Les deux nœuds reçoivent la pile Internet (IPv4). La preuve de fonctionnement retenue est double :
(i) une séquence \texttt{ping} (ICMP Echo Request/Reply) de la STA vers l’AP, et
(ii) des captures PCAP permettant une vérification externe via Wireshark/tshark.
Le ping démarre à \SI{1}{\second} afin de laisser le temps à l’association Wi-Fi et à l’initialisation des interfaces.

\section{Implémentation : décisions clés et justification}
\subsection{Paramétrage et reproductibilité}
Le scénario expose cinq paramètres (\texttt{ssid}, \texttt{simTime}, \texttt{distance}, \texttt{pcap}, \texttt{outDir}).
Cette séparation \emph{code stable} / \emph{conditions expérimentales variables} est essentielle :
elle permet de relancer le même scénario sans modification du code et prépare l’automatisation.

\subsection{Traces PCAP : rôle et valeur}
La capture PCAP n’est pas un « bonus » mais un outil de validation :
elle permet d’observer les trames de gestion (association) et les paquets ICMP.
Contrairement à une simple sortie console, le PCAP constitue une preuve audit-able et re-analysable.

\section{Exécution et résultats}
\subsection{Commande de lancement}
Exemple de lancement (adapté à la syntaxe \texttt{./ns3} de cette version) :
\begin{lstlisting}[language=bash]
./ns3 run "scratch/p1_minimal_wifi -- \
  --ssid=wifi-demo --distance=5 --simTime=10 \
  --pcap=true --outDir=/chemin/vers/results/p1"
\end{lstlisting}

\subsection{Extrait de validation Ping (à insérer)}
Coller ci-dessous un extrait représentatif (5 à 10 lignes) de la sortie ping :
\begin{lstlisting}
[COLLER ICI 5-10 LIGNES DE SORTIE PING]
\end{lstlisting}

\subsection{Traces PCAP produites}
Les captures suivantes ont été générées :
\begin{lstlisting}
results/p1/raw/wifi_wifi-demo_d5m_ap-0-0.pcap
results/p1/raw/wifi_wifi-demo_d5m_sta-1-0.pcap
\end{lstlisting}

\subsection{Analyse qualitative}
Le succès du ping confirme simultanément :
\begin{enumerate}
  \item l’association entre STA et AP (sinon absence totale de trafic IP),
  \item la cohérence de l’adressage IPv4 et l’activation des interfaces,
  \item le passage effectif de trafic applicatif sur la liaison Wi-Fi.
\end{enumerate}
L’analyse PCAP complète cette validation en rendant visible la séquence de démarrage :
trames de gestion, puis requêtes/réponses ICMP. Cette approche limite le risque de \enquote{faux positif}
où l’on conclurait à un succès sans preuve exploitable.

\section{Limites}
Ce part~1 ne vise pas la performance (débit, latence sous charge, équité multi-utilisateurs).
La topologie à deux nœuds et un flux ICMP sert avant tout à établir une base fiable et reproductible.
Les métriques de performance seront pertinentes seulement après l’introduction de trafic de charge (UDP/TCP),
de plusieurs stations, et de scénarios plus complexes.

\section{Conclusion}
Le scénario minimal AP/STA sous ns-3 est fonctionnel, paramétrable et reproductible.
Les preuves de réussite incluent (i) les réponses ICMP et (ii) des captures PCAP correctement rangées.
Cette base constitue un point de départ robuste pour les extensions ultérieures.

\end{document}
