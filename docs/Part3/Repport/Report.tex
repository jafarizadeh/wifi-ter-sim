\documentclass[a4paper,12pt]{article}
\usepackage[utf8]{inputenc}
\usepackage[T1]{fontenc}
\usepackage{graphicx}
\usepackage{geometry}
\usepackage{hyperref}
\usepackage{amsmath, amssymb, amsfonts}
\usepackage{booktabs}
\usepackage{subcaption}
\usepackage{cite}
\usepackage{float}
\usepackage{fancyhdr}

% Page layout configuration
\geometry{top=2.5cm, bottom=2.5cm, left=2.5cm, right=2.5cm}
\pagestyle{fancy}
\fancyhf{}
\rhead{\small Analysis of Wi-Fi 6 Performance}
\lhead{\small Network Simulation}
\cfoot{\thepage}

\title{\textbf{\Large Advanced Performance Analysis of IEEE 802.11ax (Wi-Fi 6)}\\[0.5cm] \large 
Effect of distance and radio channel:\\
RSSI/SNR (bonus) and performance (goodput/RTT)}
\author{JAFARIZADEH Mehdi}
\date{\today}

\begin{document}

\maketitle

\begin{abstract}
\noindent This study provides a rigorous analysis of the performance characteristics of an IEEE 802.11ax (Wi-Fi 6) network under realistic indoor propagation conditions. Utilizing the \textit{ns-3} discrete-event network simulator, we modeled a stochastic channel incorporating Log-Distance path loss ($\gamma=3.0$), Log-Normal Shadowing ($\sigma=5.0$ dB), and Nakagami-$m$ fading. The research systematically evaluates the impact of signal-to-noise ratio (SNR) degradation on Application Layer Throughput (Goodput) and Round Trip Time (RTT) across a distance sweep of 1 to 30 meters. A key finding of this research is the divergent latency behavior of UDP and TCP at the cell edge: while UDP induces bufferbloat leading to increased latency, TCP's congestion control mechanism effectively mitigates queuing delay, paradoxically reducing RTT in lossy conditions.
\end{abstract}

\tableofcontents
\newpage

\section{1. Introduction}
The IEEE 802.11ax standard, marketed as Wi-Fi 6, introduces significant improvements in spectral efficiency through technologies like OFDMA and 1024-QAM. However, the theoretical limits of these technologies are heavily constrained by physical layer (PHY) impairments in real-world environments. Accurate network planning requires understanding how large-scale fading (path loss, shadowing) and small-scale fading (multipath) affect upper-layer protocols.

This part aims to bridge the gap between ideal theoretical models and realistic performance by simulating a "digital twin" of a cluttered indoor environment. We specifically analyze the non-linear relationship between distance and Quality of Service (QoS) metrics for both connection-oriented (TCP) and connectionless (UDP) protocols.

\section{2. Theoretical Framework and Mathematical Models}
To ensure high fidelity in simulation, deterministic models were replaced with stochastic processes representing physical wave propagation.

\subsection{2.1 Large-Scale Fading: Path Loss and Shadowing}
The received power $P_r(d)$ at distance $d$ is modeled using the Log-Distance Path Loss model combined with Log-Normal Shadowing. The path loss $L(d)$ in decibels (dB) is expressed as:

\begin{equation}
    L(d) = L(d_0) + 10 \gamma \log_{10}\left(\frac{d}{d_0}\right) + X_\sigma
\end{equation}

Where:
\begin{itemize}
    \item $L(d_0)$: Reference path loss at distance $d_0 = 1$ m (calculated as 46.67 dB for 5 GHz).
    \item $\gamma$: Path Loss Exponent. We selected $\gamma = 3.0$ to model a shadowed indoor environment (heavier attenuation than free space $\gamma=2.0$).
    \item $X_\sigma$: A zero-mean Gaussian random variable $X_\sigma \sim \mathcal{N}(0, \sigma^2)$ representing Shadowing, with $\sigma = 5.0$ dB.
\end{itemize}

This stochastic term $X_\sigma$ accounts for the variation in received power due to obstacles (walls, furniture), making the channel non-deterministic.

\subsection{2.2 Small-Scale Fading: Nakagami Model}
To simulate multipath fading (rapid fluctuations in signal amplitude), we employed the Nakagami-$m$ distribution. The probability density function (PDF) of the received signal amplitude $r$ is given by:

\begin{equation}
    f(r; m, \Omega) = \frac{2m^m}{\Gamma(m)\Omega^m} r^{2m-1} \exp\left(-\frac{m}{\Omega}r^2\right)
\end{equation}

Where $m$ is the shape factor (fading severity) and $\Omega$ is the average power. In our simulation, $m$ varies with distance to model the transition from Line-of-Sight (LOS) to Non-LOS conditions.

\subsection{2.3 Throughput and Little's Law}
We define Goodput ($G$) over the useful simulation time as:
\begin{equation}
    G = \frac{8 \sum B_{rx}}{T_{sim} - T_{start}}
\end{equation}
Where $\sum B_{rx}$ is total bytes received.

For latency analysis, we refer to **Little's Law** from Queueing Theory, which states $L = \lambda W$, where $L$ is the average number of items in the queue, $\lambda$ is the arrival rate, and $W$ is the average wait time. This explains the Bufferbloat phenomenon observed in UDP tests.

\section{3. Simulation Setup}
The simulation was conducted using \textbf{ns-3 (Network Simulator 3)}. Key parameters are summarized in Table 1.

\begin{table}[H]
\centering
\caption{Simulation Parameters}
\begin{tabular}{ll}
\toprule
\textbf{Parameter} & \textbf{Value} \\
\midrule
Standard & IEEE 802.11ax (Wi-Fi 6) \\
Frequency Band & 5 GHz (Channel Width 20/40 MHz) \\
Tx Power & 16 dBm \\
Noise Figure & 7 dB \\
Rate Adaptation & Minstrel-HT \\
AQM Discipline & FqCoDel (Flow Queue CoDel) \\
Transport Protocols & UDP (OnOff), TCP (BulkSend) \\
Packet Size & 1200 Bytes \\
Simulation Distance & Sweep $\{1, 5, 10, 15, 20, 25, 30\}$ meters \\
Runs per Distance & 5 (Distinct RNG Seeds) \\
\bottomrule
\end{tabular}
\end{table}

\newpage
\section{4. Results and Critical Analysis}

\subsection{4.1 Throughput Analysis: The SNR Threshold}
Figure \ref{fig:goodput} illustrates the Goodput performance.

\begin{figure}[H]
    \centering
    \includegraphics[width=0.9\textwidth]{../../../results/p3/plots/goodput_vs_distance.png}
    \caption{Goodput vs. Distance. Shaded areas denote standard deviation ($\pm 1\sigma$).}
    \label{fig:goodput}
\end{figure}

\subsubsection*{Interpretation:}
The results show a distinct non-linear behavior characterized by two regions:
\begin{enumerate}
    \item \textbf{Saturation Region ($d < 20$m):} The channel SNR is sufficiently high to support high-order modulations (e.g., 256-QAM or 1024-QAM). The throughput is limited not by the channel, but by protocol overhead (ACKs, headers). UDP achieves $\approx 6.6$ Mbps, nearing the effective limit for the configured flow.
    
    \item \textbf{The "Waterfall" Region ($d \geq 25$m):} At 30m, we observe a sharp drop in UDP throughput ($\approx 25\%$ loss) and a significant increase in variance. This confirms the efficacy of the Shadowing model: at the cell edge, random attenuation ($\pm 5$ dB) determines whether a packet is decodable or lost. The Minstrel algorithm reacts by downgrading the MCS index, sacrificing throughput for reliability.
\end{enumerate}

\subsection{4.2 Latency Analysis: Open-Loop vs. Closed-Loop Control}
Figure \ref{fig:rtt} presents the RTT measurements, revealing a fundamental divergence between TCP and UDP.

\begin{figure}[H]
    \centering
    \includegraphics[width=0.9\textwidth]{../../../results/p3/plots/rtt_vs_distance.png}
    \caption{Mean RTT vs. Distance.}
    \label{fig:rtt}
\end{figure}

\subsubsection*{Mathematical Explanation of Divergence:}
\begin{itemize}
    \item \textbf{UDP (Open-Loop):} The application injects packets at a constant rate $\lambda_{in} = 10$ Mbps. At $d=30$m, the effective service rate of the channel $\mu$ drops below $\lambda_{in}$ ($\mu < \lambda$). According to queueing theory, the queue length $L$ grows indefinitely (until buffer overflow), increasing the waiting time $W$. This is **Bufferbloat**.
    
    \item \textbf{TCP (Closed-Loop):} TCP employs a congestion control loop. The sending rate $\lambda_{tcp}(t)$ is a function of packet loss probability $p$.
    \begin{equation}
        \lambda_{tcp} \approx \frac{MSS}{RTT \sqrt{p}}
    \end{equation}
    As distance increases, fading causes packet loss ($p \uparrow$). Consequently, TCP reduces $\lambda_{tcp}$ to match the bottleneck capacity ($\lambda_{tcp} \approx \mu$). This prevents queue buildup ($L \approx 0$), resulting in minimal queuing delay and surprisingly lower RTT.
\end{itemize}

\subsection{4.3 Temporal Stability Analysis}
The stochastic nature of the channel is best visualized in the time domain (Figure \ref{fig:stability}).

\begin{figure}[H]
    \centering
    \begin{subfigure}{0.85\textwidth}
        \centering
        \includegraphics[width=\linewidth]{../../../results/p3/plots/timeseries_d30m.png}
        \caption{Distance = 30m (Cell Edge)}
    \end{subfigure}
    \caption{Instantaneous Throughput (UDP) at Cell Edge.}
    \label{fig:stability}
\end{figure}

At 30m, the throughput exhibits binary "On/Off" behavior. The periods of zero throughput correspond to deep fading events where the instantaneous received power falls below the receiver sensitivity threshold (e.g., $-82$ dBm). This proves that the simulation correctly captures the temporal correlation of the fading channel.

\section{5. Conclusion}
This research demonstrates that realistic wireless network simulation requires more than simple deterministic models. By integrating Log-Normal Shadowing and Nakagami Fading, we showed that:
\begin{enumerate}
    \item \textbf{Coverage is probabilistic, not geometric:} The "range" of an AP is not a fixed circle but a transition zone defined by outage probability.
    \item \textbf{Transport Protocol Interaction:} The interaction between physical layer errors and transport layer flow control is critical. While UDP maximizes throughput at the cost of latency (Bufferbloat), TCP sacrifices throughput to maintain low latency and reliability.
\end{enumerate}
These findings underscore the necessity of Active Queue Management (AQM) and robust link adaptation in next-generation Wi-Fi deployments.

\end{document}